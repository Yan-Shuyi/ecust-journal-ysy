\begin{comment} % 多行注释
	
% sections/introduction.tex
% 默认引言环境
\section*{引言}
\addcontentsline{toc}{section}{引言} % 添加引言至目录
Montgomery算法自1985年提出以来,已成为模运算领域的重要算法之一。该算法通过巧妙的数学变换,将昂贵的模运算转化为一系列乘法和移位操作,显著提高了计算效率。然而,传统的Montgomery算法主要应用于整数模运算,在有限域上的扩展研究相对较少。

近年来,随着密码学和编码理论的快速发展,有限域运算在加密算法、纠错码等领域的需求日益增长。特别是在椭圆曲线密码、Reed-Solomon编码等应用中,高效的大规模有限域运算成为关键技术瓶颈。

本文在前人研究的基础上,针对有限域$\mathrm{GF}(p^k)$上的模乘运算,提出了基于Montgomery算法的扩展方案。通过构造同构映射和优化REDC算法,实现了有限域上的高效模乘运算,为相关领域的应用提供了新的解决方案。

\end{comment}

% sections/introduction.tex
% 使用自定义引言环境

\begin{introduction}
	Montgomery算法自1985年提出以来,已成为模运算领域的重要算法之一。该算法通过巧妙的数学变换,将昂贵的模运算转化为一系列乘法和移位操作,显著提高了计算效率。然而,传统的Montgomery算法主要应用于整数模运算,在有限域上的扩展研究相对较少。
	
	近年来,随着密码学和编码理论的快速发展,有限域运算在加密算法、纠错码等领域的需求日益增长。特别是在椭圆曲线密码、Reed-Solomon编码等应用中,高效的大规模有限域运算成为关键技术瓶颈。
	
	本文在前人研究的基础上,针对有限域$\mathrm{GF}(p^k)$上的模乘运算,提出了基于Montgomery算法的扩展方案。通过构造同构映射和优化REDC算法,实现了有限域上的高效模乘运算,为相关领域的应用提供了新的解决方案。
\end{introduction}
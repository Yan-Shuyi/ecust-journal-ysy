% sections/results.tex

\section{引用示例}

% ==================== 基本定义部分 ====================
\subsection{基本定义}

\begin{definition}
	设 $\mathrm{GF}(p^k)$ 是特征为 $p$ 的有限域,其同构映射定义为:
	\begin{equation}
		\label{eq:isomorphism} % 公式标签,用于交叉引用
		% 标签命名规范:eq:描述性名称
		% 例如:eq:isomorphism, eq:matrix, eq:algorithm
		\phi: \mathrm{GF}(p^k) \rightarrow \mathrm{GF}(p)[x]/(f(x))
	\end{equation}
	其中 $f(x)$ 是 $\mathrm{GF}(p)$ 上的 $k$ 次不可约多项式。
	
	% 说明:
	% \mathrm{GF}(p^k) - 有限域表示法
	% \rightarrow - 映射符号
	% \mathrm{GF}(p)[x]/(f(x)) - 多项式商环表示
\end{definition}

% ==================== 公式引用示例 ====================
\subsection{公式引用示例}

% 使用 \eqref{} 引用公式(自动添加括号)
根据同构映射定义(公式 \eqref{eq:isomorphism}),我们可以将有限域运算转化为多项式环运算。

% 使用 \ref{} 引用公式(手动添加括号)
基于公式 \ref{eq:isomorphism} 的数学基础,我们提出了改进的有限域乘法方法。

% 多公式引用示例
% 假设有多个公式时,可以这样引用:
% 如公式 \eqref{eq:formula1}、\eqref{eq:formula2} 和 \eqref{eq:formula3} 所示

% ==================== 其他引用类型示例 ====================
% 以下为其他类型的引用示例(注释状态,可根据需要启用)

% 图片引用示例
% 如图 \ref{fig:performance} 所示,算法性能有显著提升。

% 表格引用示例  
% 详细数据参见表 \ref{tab:experiment_results}。

% 章节引用示例
% 关于理论基础,请参考第\ref{sec:theory}节。

% 参考文献引用示例
% Montgomery算法\cite{montgomery1985}在密码学中广泛应用。

% ==================== 编译说明 ====================
% 重要:需要编译两次才能正确显示交叉引用!
% 第一次编译:xelatex main.tex(收集标签信息)
% 第二次编译:xelatex main.tex(生成正确编号)

% ==================== 标签命名规范 ====================
% 公式标签:eq:description    (如:eq:isomorphism)
% 图片标签:fig:description   (如:fig:performance)
% 表格标签:tab:description   (如:tab:results)
% 章节标签:sec:description   (如:sec:introduction)
% 定理标签:thm:description   (如:thm:main)

% ==================== 高级用法示例 ====================
% 多公式环境中的引用
% \begin{align}
	%     \label{eq:matrix_form}
	%     \mathbf{A} &= \begin{bmatrix}
		%         a_{11} & a_{12} \\
		%         a_{21} & a_{22}
		%     \end{bmatrix} \\
	%     \label{eq:matrix_property}
	%     \det(\mathbf{A}) &= a_{11}a_{22} - a_{12}a_{21}
	% \end{align}
% 矩阵定义如公式 \eqref{eq:matrix_form} 所示,其行列式计算如公式 \eqref{eq:matrix_property} 所示。

% ==================== 错误排查提示 ====================
% 如果引用显示为??,请检查:
% 1. 是否编译了两次
% 2. 标签名称是否拼写正确
% 3. 标签是否在相应环境内定义
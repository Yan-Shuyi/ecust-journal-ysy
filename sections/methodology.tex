% sections/methodology.tex

\section{部分基础知识}

Montgomery算法是一种能够加速模乘运算的算法,通过对乘法的优化,能够显著提高计算效率。本文将对Montgomery算法进行扩展,以适应有限域的运算需求。在本节中我们将介绍一下有限域和Montgomery算法及其需要的基础知识。

\subsection{环的部分性质}

有限域,也被称为伽罗瓦域(Galois field),是仅含有限个元素的域。这一概念是由伽罗瓦(Galois, E.)在19世纪30年代研究代数方程根式求解问题时提出的。有限域在近代编码、计算机理论、组合数学等方面有着广泛的应用。这里仅给出部分定义及其性质:

\begin{definition}\label{def:characteristic}
	一个无零因子环$R$的非零元的相同的(对加法来说的)阶叫做环$R$的特征\cite{zhang1978}。
\end{definition}

\begin{property}\label{prop:char-prime}
	如果无零因子环$R$的特征是有限整数$n$,那么$n$是一个素数。
\end{property}

\begin{proof}
	假如$n$不是素数:$n = n_1 n_2$,那么对于$R$中的一个不等于零的元$a$,$n_1 a \neq 0$,$n_2 a \neq 0$,但$(n_1 a)(n_2 a) = (n_1 n_2) a^2 = 0$,这与$R$没有零因子的假定冲突。\qedhere
\end{proof}

因为无零因子的交换幺环是整环,我们知道域是一个整环,所以根据性质\ref{prop:char-prime}我们得到以下推论:

\begin{property}\label{prop:field-char}
	整环以及域的特征或是无限大,或是一个素数$p$。
\end{property}

\subsection{Montgomery模乘算法}

彼得·L·蒙哥马利(Peter L. Montgomery)在1985年提出了一种在避免模除的情况下,计算两个模$N$($N>1$)整数(即模$N$剩余类)相乘的方法\cite{montgomery1985}。其中加法和减法不变,该方法在对进行多次模$N$运算时可以提高效率。

\subsubsection{算法核心}

Montgomery通过REDC算法,将模乘运算进行转化,使其可以通过一系列乘法和加法运算来完成。

设$N$是一个大于1的整数。我们选择一个与$N$互质的基数$R$,其中$R > N$(这个$R$的选择应该尽可能地使得模$R$的计算成本低)。然后设$R^{-1}$和$N'$是满足$0 < R^{-1} < N$和$0 < N' < R$和$RR^{-1} - NN' = 1$的整数。

对于每个$i$,$0 \leq i < N$,我们让$\bar{i}$来表示包含$iR^{-1} \mod N$的剩余类,也就是所有可以写成$iR^{-1} + mN$形式的整数的集合。这其实是一个完全剩余系。使用这种新的剩余系的原因是我们能够从$T$($0 \leq T < RN$)快速计算$TR^{-1} \mod N$,如算法\ref{alg:redc}所示:

\begin{algorithm}
	\caption{REDC算法}\label{alg:redc}
	\begin{algorithmic}[1]
		\Procedure{REDC}{$T$}
		\State $m \gets (T \bmod R) \cdot N' \bmod R$ \Comment{因此 $0 \leq m < R$}
		\State $t \gets (T + mN) / R$
		\If {$t \geq N$}
		\State \Return $t - N$
		\Else
		\State \Return $t$
		\EndIf
		\EndProcedure
	\end{algorithmic}
\end{algorithm}

这个算法返回的$t$即$TR^{-1} \mod N$。

如果我们给定两个在$0$和$N-1$之间的整数$x$和$y$,设$z = \text{REDC}(xy)$,也就是$z \equiv (xy) R^{-1} \mod N$。等式两边同乘$R^{-1}$有$(xR^{-1})(yR^{-1}) \equiv zR^{-1} \mod N$。因为REDC算法给出的$z < N$,所以在新的剩余系下$z$是$x$和$y$的乘积,即$\bar{z} = \bar{x}\bar{y}$。